\section*{Exercise 51}
New: 
\newline
\begin{enumerate}
    \item Syntax to use new operator: To allocate memory of any data type, for instance: int *p = new int; It is appropriately used when you want to move around pointers rather than the object.
    \item Allocate block of memory: new operator is also used to allocate a block(an array) of memory of a certain data-type. for example: int *p = new int[10]; It is appropriately used when you want to use the objects but the size is unknown. 
    \item Initializes a certain data-type at a given address: It returns the address of the initialized data-type. For instance: int *p = new(address) int; It is appropriately used when you need to initialize part of a large amount of memory with a new object.
    \item Initializes a certain size data-type at a given address: It returns the first element of the array at the location. For example: int *p = new(address) int[10]; It is appropriately used when you need to initialize part of a large amount of memory with an array of new objects.
\end{enumerate}
With every variant of operator new we need a corresponding delete.
\newline
Delete: 
\begin{enumerate}
    \item delete p; it deletes the memory previously allocated with int
    \item delete[] p; it deletes all the objects where the pointer p pointed to, as well as the dynamically allocated array previously containing the objects.
    \item For 3 and 4 there is no point for a delete function for this as the operator doesn't create any memory, you can simple call a destructor function to delete the data-type you put there. Integer here is not a good example, so we will demonstrate with a string object. 
    \begin{lstlisting}
        for (size_t idx = 0; idx != size; ++idx)
        p[idx].~string();
    \end{lstlisting}
\end{enumerate}
