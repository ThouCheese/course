\documentclass{article}
\usepackage[utf8]{inputenc}
\usepackage[T1]{fontenc}
\usepackage[utf8]{inputenc}
\usepackage{lmodern}
\usepackage[a4paper, margin=1in]{geometry}

\usepackage{minted}
\large
\title{C++ Assignment 3}
\begin{document}
\begin{titlepage}
    \begin{center}
        \line(1,0){300}\\
        [0.65cm]
        \huge{\bfseries Assignment III}\\
        \line(1,0){300}\\
        \textsc{\Large C++ Course Part I}\\
        \textsc{\LARGE \today}\\
        [5.5cm]     
    \end{center}
    \begin{flushright}
        \textsc{\Large L. Wester\\S2755351}\\
        [0.5cm]
    \end{flushright}
\end{titlepage}
\section*{Exercise 50}
\subsection*{/main.ih}
\begin{minted}[frame=lines, linenos, fontsize=\large]{c++}
#include "charcount/charcount.h"
#include <iostream>

using namespace std;

void showChar(char ch);
\end{minted}
\subsection*{/main.cpp}
\begin{minted}[frame=lines, linenos, fontsize=\large]{c++}
#include "main.ih"

int main()
{
    CharCount counter;
    counter.count(std::cin);
    for (size_t index = 0; index != counter.info().nChar; ++index)
    {
        showChar(counter.info().ptr[index].ch);
        cout << " found " << counter.info().ptr[index].count << " times\n";
    }
    cout << "capacity was " << counter.capacity() << '\n';
}

\end{minted}
\subsection*{/show\_char.cpp}
\begin{minted}[frame=lines, linenos, fontsize=\large]{c++}
#include "main.ih"

void showChar(char ch)
{
    cout << '\'';
    switch (ch) {
        case '\t':
            cout << "\\t";
        break;
        case '\n':
            cout << "\\n";
        break;
        default:
            cout << ch;
    }
    cout << '\'';
}
\end{minted}
\subsection*{/char/char.ih}
\begin{minted}[frame=lines, linenos, fontsize=\large]{c++}
#include "char.h"
#include <iostream>

using namespace std;

\end{minted}
\subsection*{/char/char.h}
\begin{minted}[frame=lines, linenos, fontsize=\large]{c++}
#ifndef CHAR_H
#define CHAR_H

#include <cstddef>

struct Char 
{
    char ch;
    size_t count = 1;

    Char() = default;
    Char(char ch);
};

#endif // CHAR_H

\end{minted}
\subsection*{/char/char1.cpp}
\begin{minted}[frame=lines, linenos, fontsize=\large]{c++}
#include "char.ih"

Char::Char(char ch)
    :
        ch(ch)
{}
\end{minted}

\subsection*{/charcount/charcount.ih}
\begin{minted}[frame=lines, linenos, fontsize=\large]{c++}
#include "charcount.h"
#include <iostream>

using namespace std;

\end{minted}
\subsection*{/charcount/charcount.h}
\begin{minted}[frame=lines, linenos, fontsize=\large]{c++}
#ifndef CHARCOUNT_H
#define CHARCOUNT_H

#include "../enums/action.h"
#include "../charinfo/charinfo.h"
#include <iostream>

class CharCount 
{
    size_t d_work_index = 0; // set by locate, used by insert and add
    size_t d_capacity = 10;
    CharInfo d_char_info;
public:
    CharCount();
    ~CharCount();
    void count(std::istream &in);
    CharInfo const &info() const;
    size_t capacity() const;

private:
    Action locate(char ch);
    void add(char ch);
    void insert(char ch);
    void append(char ch);
    Char *raw_capacity(size_t capacity) const; // leaky!
    void enlarge();
};

#endif // CHARCOUNT_H

\end{minted}
\subsection*{/charcount/add.cpp}
\begin{minted}[frame=lines, linenos, fontsize=\large]{c++}
#include "charcount.ih"

// takes the Char struct at d_work_index, and increments its count
void CharCount::add(char ch)
{
    ++d_char_info.ptr[d_work_index].count;
}
\end{minted}
\subsection*{/charcount/append.cpp}
\begin{minted}[frame=lines, linenos, fontsize=\large]{c++}
#include "charcount.ih"

void CharCount::append(char ch)
{
    if (++d_char_info.nChar == d_capacity)
        enlarge();
    d_char_info.ptr[d_char_info.nChar - 1] = Char(ch);
}

\end{minted}
\subsection*{/charcount/capacity.cpp}
\begin{minted}[frame=lines, linenos, fontsize=\large]{c++}
#include "charcount.ih"

size_t CharCount::capacity() const
{
    return d_capacity;
}
\end{minted}
\subsection*{/charcount/charcount1.cpp}
\begin{minted}[frame=lines, linenos, fontsize=\large]{c++}
#include "charcount.ih"

CharCount::CharCount()
    :
        d_char_info()
{
    d_char_info.ptr = raw_capacity(d_capacity); 
}
\end{minted}
\subsection*{/charcount/count.cpp}
\begin{minted}[frame=lines, linenos, fontsize=\large]{c++}
#include "charcount.ih"

void CharCount::count(std::istream &in)
{
    char ch;
    cin >> noskipws;

    static void (CharCount::*fn_list[3])(char) = 
        {&CharCount::insert, &CharCount::append, &CharCount::add};

    while (cin >> ch)
    {
        Action action = locate(ch);
        (this->*fn_list[action])(ch);
    }
}

\end{minted}
\subsection*{/charcount/destructor.cpp}
\begin{minted}[frame=lines, linenos, fontsize=\large]{c++}
#include "charcount.ih"

CharCount::~CharCount()
{
    for (size_t index = 0; index != d_char_info.nChar; ++index)
        d_char_info.ptr[index].~Char();
    operator delete(d_char_info.ptr);
}
\end{minted}
\subsection*{/charcount/enlarge.cpp}
\begin{minted}[frame=lines, linenos, fontsize=\large]{c++}
#include "charcount.ih"

void CharCount::enlarge()
{
    Char *old = d_char_info.ptr;
    d_char_info.ptr = raw_capacity(d_capacity * 2);

    for (size_t index = 0; index != d_capacity; ++index)
    {
        d_char_info.ptr[index] = old[index];
        old[index].~Char();
    }
    operator delete(old);
    d_capacity *= 2;
}
\end{minted}
\subsection*{/charcount/info.cpp}
\begin{minted}[frame=lines, linenos, fontsize=\large]{c++}
#include "charcount.ih"

CharInfo const &CharCount::info() const
{
    return d_char_info;
}
\end{minted}
\subsection*{/charcount/insert.cpp}
\begin{minted}[frame=lines, linenos, fontsize=\large]{c++}
#include "charcount.ih"

void CharCount::insert(char ch)
{
    if (++d_char_info.nChar == d_capacity)
        enlarge();
    for (size_t index = d_capacity; --index != d_work_index;)
        d_char_info.ptr[index] = d_char_info.ptr[index - 1];
    d_char_info.ptr[d_work_index] = Char(ch);
}
\end{minted}
\subsection*{/charcount/locate.cpp}
\begin{minted}[frame=lines, linenos, fontsize=\large]{c++}
#include "charcount.ih"

Action CharCount::locate(char ch)
{
    if (not d_char_info.ptr)
        return Action::APPEND;
    for (size_t index = 0; index != d_char_info.nChar; ++index)
    {
        char cur_char = d_char_info.ptr[index].ch;
        if (cur_char == ch) // set d_work_index to the right value, return ADD
            return d_work_index = index, Action::ADD;
        else if (cur_char > ch) // same but with INSERT
            return d_work_index = index, Action::INSERT;
    }
    return Action::APPEND;
}
\end{minted}
\subsection*{/charcount/raw\_capacity.cpp}
\begin{minted}[frame=lines, linenos, fontsize=\large]{c++}
#include "charcount.ih"

Char *CharCount::raw_capacity(size_t capacity) const
{
    return static_cast<Char *>(operator new(sizeof(CharCount) * capacity));
}
\end{minted}

\subsection*{/charinfo/charinfo.ih}
\begin{minted}[frame=lines, linenos, fontsize=\large]{c++}
#include "charinfo.h"
#include <iostream>

using namespace std;

\end{minted}
\subsection*{/charinfo/charinfo.h}
\begin{minted}[frame=lines, linenos, fontsize=\large]{c++}
#ifndef CHARINFO_H
#define CHARINFO_H

#include "../char/char.h"
#include "../enums/action.h"

struct CharInfo 
{
    size_t nChar = 0;
    Char *ptr = 0;
};

#endif // CHARINFO_H
\end{minted}

\subsection*{/enums/action.h}
\begin{minted}[frame=lines, linenos, fontsize=\large]{c++}
#ifndef ACTION_H
#define ACTION_H

// these are used as array indices, do not modify their values
enum Action
{
    INSERT = 0,
    ADD = 1,
    APPEND = 2,
};

#endif // ACTION_H
\end{minted}


\section*{Exercise 51}
New: 
\newline
\begin{enumerate}
    \item new / new[]: To allocate memory of any data type, for instance: int *p = new int; Used when we need a pointer to persistent data, that does not get destroyed as the position that created it goes out of scope. When used to create a pointer to data this variant calls the constructor on the/each element it allocates, unlike:
    \item new(placement) / new(placement)[]: Takes an already allocated piece of memory and initializes it by calling the default constructor. If the memory was already initialized, the previous data is overwritten.
    \item operator new / operator new[]: Allocates a new piece of memory without initializing it.
\end{enumerate}
With every variant of operator new we need a corresponding delete.
\newline
Delete: 
\begin{enumerate}
    \item delete p / delete p[]: delete all memory and return it to the os.
    \item there is not corresponding deletion operator to new(placement), as it does not allocate memory. Instead, call the destructor.
    \item operator delete / operator delete[]: returns the referred memory to the os without calling the destructor.
\end{enumerate}


\section*{Exercise 52}
\subsection*{/strings/strings.h}
\begin{minted}[frame=lines, linenos, fontsize=\large]{c++}
#ifndef STRINGS_H
#define STRINGS_H

#include "../stringsdata/stringsdata.h"
#include <string>
#include <istream>

class Strings 
{
    size_t d_size;                     // number of stored strings
    std::string *d_str;                // pointer to `d_str` strings
    std::string d_empty;
public:
    Strings();
    Strings(std::istream &in);
    Strings(char **env);
    Strings(int argc, char *argv[]);
    ~Strings();
    size_t size() const;
    std::string *data();
    std::string const &at(size_t index) const;
    std::string &at(size_t index);
    StringsData release(); // danger! leaks!
    void add(std::string the_next_string);

private:
    std::string &at_backdoor(size_t index) const;
    void realloc(size_t new_size);
};

#endif // STRINGS_H

\end{minted}
\subsection*{/strings/destructor.cpp}
\begin{minted}[frame=lines, linenos, fontsize=\large]{c++}
#include "strings.ih"

Strings::~Strings()
{
    delete[] d_str;
}
\end{minted}


\section*{Exercise 53}
\subsection*{/main.ih}
\begin{minted}[frame=lines, linenos, fontsize=\large]{c++}
#include "strings/strings.h"
#include <iostream>

extern char **environ;

using namespace std;

\end{minted}
\subsection*{/main.cpp}
\begin{minted}[frame=lines, linenos, fontsize=\large]{c++}
#include "main.ih"

int main()
{
    Strings str(environ);
    size_t size = str.size();
    for (size_t index = 0; index != size; ++index)
        cout << str.at(index) << '\n';
}
\end{minted}
\subsection*{/filter/filter.ih}
\begin{minted}[frame=lines, linenos, fontsize=\large]{c++}
#include "filter.h"
#include <iostream>

using namespace std;
\end{minted}
\subsection*{/filter/filter.h}
\begin{minted}[frame=lines, linenos, fontsize=\large]{c++}
#ifndef FILTER_H
#define FILTER_H

#include "../strings/strings.h"

class Filter 
{
    Strings const &d_strings; // store a reference because a value is woefully inefficient
    size_t d_index = 0;
public:
    Filter(Strings const &strings);
    void display();
private:
    bool getline(std::string &line);
};

#endif // FILTER_H
\end{minted}
\subsection*{/filter/display.cpp}
\begin{minted}[frame=lines, linenos, fontsize=\large]{c++}
#include "filter.ih"

void Filter::display()
{
    size_t line_num = 0;
    string line;
    while (true)
    {
        if (not this->getline(line))
            return; // all lines were empty
        if (line.find_first_not_of(" \t") != string::npos) // non empty
            break;
    }
    cout << line << '\n';

    size_t n_empty = 0; // number of empty lines
    while (this->getline(line))
        if (line.find_first_not_of(" \t") == string::npos)
            ++n_empty;
        else
        {
            cout << string(n_empty, '\n') << line << '\n';
            n_empty = 0;
        }
}
\end{minted}
\subsection*{/filter/filter1.cpp}
\begin{minted}[frame=lines, linenos, fontsize=\large]{c++}
#include "filter.ih"

Filter::Filter(Strings const &strings)
    :
        d_strings(strings)
{}
\end{minted}

\subsection*{/strings/strings.ih}
\begin{minted}[frame=lines, linenos, fontsize=\large]{c++}
#include "strings.h"
#include <iostream>

using namespace std;

\end{minted}
\subsection*{/strings/strings.h}
\begin{minted}[frame=lines, linenos, fontsize=\large]{c++}
#ifndef STRINGS_H
#define STRINGS_H

#include "../stringsdata/stringsdata.h"
#include <string>
#include <istream>

class Strings 
{
    size_t d_size = 0;                     // number of stored strings
    size_t d_capacity = 0;
    std::string **d_str = 0;               // pointer to `d_str` *string's
public:
    Strings();
    Strings(std::istream &in);
    Strings(char **env);
    Strings(int argc, char *argv[]);
    ~Strings();
    size_t size() const;
    std::string const &at(size_t index) const;
    std::string &at(size_t index);
    StringsData release(); // danger! leaks!
    void add(std::string const &the_next_string);

    size_t capacity() const;

private:
    // private backdoor
    std::string **data();
    // replaces the body of `Strings(char **env)` for reuse
    void init(int argc, char **argv);
    void resize(size_t capacity);
    void destroy();
    void ensure_capacity();

    std::string &at_backdoor(size_t index) const;
    void realloc(size_t new_size);
    
    static std::string **raw_pointers(size_t capacity);
};

#endif // STRINGS_H

\end{minted}
\subsection*{/strings/add.cpp}
\begin{minted}[frame=lines, linenos, fontsize=\large]{c++}
#include "strings.ih"

void Strings::add(string const &the_next_string)
{
    ensure_capacity();
    d_str[d_size] = new string(the_next_string);
    ++d_size;
}
\end{minted}
\subsection*{/strings/at.cpp}
\begin{minted}[frame=lines, linenos, fontsize=\large]{c++}
#include "strings.ih"

string &Strings::at(size_t index)
{
    return at_backdoor(index);
}
\end{minted}
\subsection*{/strings/at\_backdoor.cpp}
\begin{minted}[frame=lines, linenos, fontsize=\large]{c++}
#include "strings.ih"

string &Strings::at_backdoor(size_t index) const
{
    // return index < d_size ? &d_str[index] : &d_empty;
    static string empty{};
    return index < d_size ? *(d_str[index]) : empty; 
}

\end{minted}
\subsection*{/strings/at\_const.cpp}
\begin{minted}[frame=lines, linenos, fontsize=\large]{c++}
#include "strings.ih"

string const &Strings::at(size_t index) const
{
    return at_backdoor(index);
}
\end{minted}
\subsection*{/strings/data.cpp}
\begin{minted}[frame=lines, linenos, fontsize=\large]{c++}
#include "strings.ih"

string **Strings::data()
{
    return d_str;
}
\end{minted}
\subsection*{/strings/destroy.cpp}
\begin{minted}[frame=lines, linenos, fontsize=\large]{c++}
#include "strings.ih"

void Strings::destroy()
{
    for (size_t index = 0; index != d_size; ++index)
        delete d_str[index];
    delete[] d_str;
}
\end{minted}
\subsection*{/strings/destructor.cpp}
\begin{minted}[frame=lines, linenos, fontsize=\large]{c++}
#include "strings.ih"

Strings::~Strings()
{
    destroy();
}
\end{minted}
\subsection*{/strings/ensure\_capacity.cpp}
\begin{minted}[frame=lines, linenos, fontsize=\large]{c++}
#include "strings.ih"

void Strings::ensure_capacity()
{
    if (d_capacity <= d_size)
        resize(d_capacity * 2);
}
\end{minted}
\subsection*{/strings/init.cpp}
\begin{minted}[frame=lines, linenos, fontsize=\large]{c++}
#include "strings.ih"

// this constructor functionality is encapuslated in a separate function to
// enable easy reuse of code in string3 and strings4
void Strings::init(int argc, char **argv)
{
    resize(argc);
    d_capacity = argc;
    for (char **ptr = argv; ptr != argv + argc; ++ptr)
        add(string(*ptr));
}
\end{minted}
\subsection*{/strings/raw\_pointers.cpp}
\begin{minted}[frame=lines, linenos, fontsize=\large]{c++}
#include "strings.ih"

string **Strings::raw_pointers(size_t capacity)
{

    return new string*[capacity];
}
\end{minted}
\subsection*{/strings/release.cpp}
\begin{minted}[frame=lines, linenos, fontsize=\large]{c++}
#include "strings.ih"

StringsData Strings::release()
{
    resize(d_size); // scratch all trailing memory
    StringsData data(d_size, d_str);
    d_str = 0;
    d_size = 0;
    d_capacity = 0;
    return data;
}
\end{minted}
\subsection*{/strings/resize.cpp}
\begin{minted}[frame=lines, linenos, fontsize=\large]{c++}
#include "strings.ih"

// allocates additional capacity, without updating d_capacity, or
// intializing it. Dangerous and only used by resize
string **reserve(string **old, string **new_mem, size_t capacity)
{
    if (old)
    {
        for (size_t index = 0; index != capacity; ++index)
            new_mem[index] = old[index];
        delete[] old;
    }
    return new_mem;
}

// reserves capacity, initializes it, then updates d_capacity
// also perfectly happy to reduce size
void Strings::resize(size_t capacity)
{
    d_str = reserve(d_str, raw_pointers(capacity), 
                    capacity > d_capacity ? capacity : d_capacity);
    for (size_t index = d_capacity; index < capacity; ++index)
        d_str[index] = new string;
    d_capacity = capacity;  
}
\end{minted}
\subsection*{/strings/size.cpp}
\begin{minted}[frame=lines, linenos, fontsize=\large]{c++}
#include "strings.ih"

size_t Strings::size() const
{
    return d_size;   
}
\end{minted}
\subsection*{/strings/strings1.cpp}
\begin{minted}[frame=lines, linenos, fontsize=\large]{c++}
#include "strings.ih"

Strings::Strings()
{
    d_capacity = 1;
    resize(1);
}

\end{minted}
\subsection*{/strings/strings2.cpp}
\begin{minted}[frame=lines, linenos, fontsize=\large]{c++}
#include "strings.ih"

Strings::Strings(istream &in)
{
    string line;
    while (getline(in, line))
        add(line);
}

\end{minted}
\subsection*{/strings/strings3.cpp}
\begin{minted}[frame=lines, linenos, fontsize=\large]{c++}
#include "strings.ih"

Strings::Strings(char **env)
{
    int amount = 0;
    while (*(env + amount))
        ++amount;
    init(amount, env);
}
\end{minted}
\subsection*{/strings/strings4.cpp}
\begin{minted}[frame=lines, linenos, fontsize=\large]{c++}
#include "strings.ih"

Strings::Strings(int argc, char *argv[])
{
    init(argc, argv);
}
\end{minted}

\subsection*{/stringsdata/stringsdata.ih}
\begin{minted}[frame=lines, linenos, fontsize=\large]{c++}
#include "stringsdata.h"
#include <iostream>

using namespace std;

\end{minted}
\subsection*{/stringsdata/stringsdata.h}
\begin{minted}[frame=lines, linenos, fontsize=\large]{c++}
#ifndef STRINGSDATA_H
#define STRINGSDATA_H

#include <cstddef>
#include <string>

struct StringsData 
{
    size_t size;
    std::string **ptr;

    StringsData(size_t size, std::string **ptr);
};

#endif // STRINGSDATA_H

\end{minted}
\subsection*{/stringsdata/stringsdata1.cpp}
\begin{minted}[frame=lines, linenos, fontsize=\large]{c++}
#include "stringsdata.ih"

StringsData::StringsData(size_t size, string **ptr)
    :
        size(size),
        ptr(ptr)
{}

\end{minted}


\section*{Exercise 54}
\subsection*{/main.ih}
\begin{minted}[frame=lines, linenos, fontsize=\large]{c++}
#include "sort/sort.ih"
#include <iostream>

using namespace std;

int increasing(string **string1, string **string2);

\end{minted}
\subsection*{/increasing.cpp}
\begin{minted}[frame=lines, linenos, fontsize=\large]{c++}
#include "main.ih"

int increasing(string const **string1, string const **string2)
{
    return (*string1)->compare(**string2);
}

\end{minted}
\subsection*{/main.cpp}
\begin{minted}[frame=lines, linenos, fontsize=\large]{c++}
#include "main.ih"

int main()
{
    string strs[] = {"hoi", "wie", "wil", "een", "string"};
    Sort sorter(&increasing);
    sorter.sort(strs, 5);
    for (string str : strs)
        cout << str << '\n';
}
\end{minted}
\subsection*{/nocasedec.cpp}
\begin{minted}[frame=lines, linenos, fontsize=\large]{c++}

\end{minted}
\subsection*{/sort/sort.ih}
\begin{minted}[frame=lines, linenos, fontsize=\large]{c++}
#include "sort.h"
#include <iostream>
#include <cstdlib>

using namespace std;

\end{minted}
\subsection*{/sort/sort.h}
\begin{minted}[frame=lines, linenos, fontsize=\large]{c++}
#ifndef SORT_H
#define SORT_H

#include <string>

class Sort
{
    int (*d_fn_ptr)(std::string**, std::string**);

public:
    Sort(int (*fn_ptr)(std::string**, std::string**));
    void sort(std::string *start, size_t count) const;
};

#endif // SORT_H


\end{minted}
\subsection*{/sort/sort.cpp}
\begin{minted}[frame=lines, linenos, fontsize=\large]{c++}
#include "sort.ih"

void Sort::sort(std::string *start, size_t count) const
{   
    qsort(start, count, sizeof(string *), 
        reinterpret_cast<int (*)(void const *, void const *)>
            (d_fn_ptr));
}
\end{minted}
\subsection*{/sort/sort1.cpp}
\begin{minted}[frame=lines, linenos, fontsize=\large]{c++}
#include "sort.ih"

Sort::Sort(int (*fn_ptr)(std::string**, std::string**))
    :
        d_fn_ptr(fn_ptr)
{}

\end{minted}


\section*{Exercise 55}
\subsection*{/main.ih}
\begin{minted}[frame=lines, linenos, fontsize=\large]{c++}
#include "strings/strings.h"
#include <iostream>

extern char **environ;

using namespace std;

\end{minted}
\subsection*{/main.cpp}
\begin{minted}[frame=lines, linenos, fontsize=\large]{c++}
#include "main.ih"

int main()
{
    // Strings str(environ);
    // for (size_t index = 0; index != str.size(); ++index)
    //     cout << str.data()[index] << '\n';
    

    for (size_t iter = 0; iter != 1000; ++iter)
    {
        Strings env(environ);

        for (size_t rept = 0; rept != 100; ++rept)
        {
            for (char **ptr = environ; *ptr; ++ptr)
                env.add(*ptr);
        }
    } 
}
\end{minted}
\subsection*{/strings/strings.ih}
\begin{minted}[frame=lines, linenos, fontsize=\large]{c++}
#include "strings.h"
#include <iostream>

using namespace std;

\end{minted}
\subsection*{/strings/strings.h}
\begin{minted}[frame=lines, linenos, fontsize=\large]{c++}
#ifndef STRINGS_H
#define STRINGS_H

#include "../stringsdata/stringsdata.h"
#include <string>
#include <istream>

class Strings 
{
    size_t d_size = 0;                    // number of stored strings
    size_t d_capacity = 0;
    std::string *d_str = 0;               // pointer to internal data
public:
    Strings();
    Strings(std::istream &in);
    Strings(char **env);
    Strings(int argc, char *argv[]);
    ~Strings();
    size_t size() const;
    std::string const &at(size_t index) const;
    std::string &at(size_t index);
    StringsData release(); // danger! leaks!
    void add(std::string the_next_string);

    size_t capacity() const;

private:
    std::string *data();

    // replaces the body of `Strings(char **env)` for reuse
    void init(int argc, char **argv);
    void reserve(size_t capacity);
    void resize(size_t capacity);
    static std::string *raw_memory(size_t capacity);
    void destroy();
    void ensure_capacity();

    std::string &at_backdoor(size_t index) const;
    void realloc(size_t new_size);
};

#endif // STRINGS_H

\end{minted}
\subsection*{/strings/add.cpp}
\begin{minted}[frame=lines, linenos, fontsize=\large]{c++}
#include "strings.ih"

void Strings::add(string the_next_string)
{
    ensure_capacity();
    new(d_str + d_size) string(the_next_string);
    ++d_size;
}

\end{minted}
\subsection*{/strings/at.cpp}
\begin{minted}[frame=lines, linenos, fontsize=\large]{c++}
#include "strings.ih"

string &Strings::at(size_t index)
{
    return at_backdoor(index);
}
\end{minted}
\subsection*{/strings/at\_backdoor.cpp}
\begin{minted}[frame=lines, linenos, fontsize=\large]{c++}
#include "strings.ih"

string &Strings::at_backdoor(size_t index) const
{
    static string empty{};
    return index < d_size ? d_str[index] : empty; 
}

\end{minted}
\subsection*{/strings/at\_const.cpp}
\begin{minted}[frame=lines, linenos, fontsize=\large]{c++}
#include "strings.ih"

string const &Strings::at(size_t index) const
{
    return at_backdoor(index);
}
\end{minted}
\subsection*{/strings/data.cpp}
\begin{minted}[frame=lines, linenos, fontsize=\large]{c++}
#include "strings.ih"

string *Strings::data()
{
    return d_str;
}
\end{minted}
\subsection*{/strings/destroy.cpp}
\begin{minted}[frame=lines, linenos, fontsize=\large]{c++}
#include "strings.ih"

void Strings::destroy()
{
    for (size_t index = 0; index != d_size; ++index)
        d_str[index].~basic_string();
    operator delete(d_str);
}
\end{minted}
\subsection*{/strings/destructor.cpp}
\begin{minted}[frame=lines, linenos, fontsize=\large]{c++}
#include "strings.ih"

Strings::~Strings()
{
    destroy();
}
\end{minted}
\subsection*{/strings/ensure\_capacity.cpp}
\begin{minted}[frame=lines, linenos, fontsize=\large]{c++}
#include "strings.ih"

void Strings::ensure_capacity()
{
    if (d_capacity <= d_size)
        resize(d_capacity * 2);
}
\end{minted}
\subsection*{/strings/init.cpp}
\begin{minted}[frame=lines, linenos, fontsize=\large]{c++}
#include "strings.ih"

void Strings::init(int argc, char **argv)
{
    resize(argc);
    d_capacity = argc;
    for (char **ptr = argv; ptr != argv + argc; ++ptr)
        add(string(*ptr));
}
\end{minted}
\subsection*{/strings/raw\_memory.cpp}
\begin{minted}[frame=lines, linenos, fontsize=\large]{c++}
#include "strings.ih"

string *Strings::raw_memory(size_t capacity)
{
    return static_cast<string *>(operator new(capacity * sizeof(string)));
}
\end{minted}
\subsection*{/strings/release.cpp}
\begin{minted}[frame=lines, linenos, fontsize=\large]{c++}
#include "strings.ih"

StringsData Strings::release()
{
    resize(d_size); // scratch all trailing memory
    StringsData data(d_size, d_str);
    d_capacity = 0;
    d_size = 0;
    reserve(1);
    d_capacity = 1;
    return data;
}
\end{minted}
\subsection*{/strings/reserve.cpp}
\begin{minted}[frame=lines, linenos, fontsize=\large]{c++}
#include "strings.ih"

// allocates additional capacity, without updating d_capacity, or
// intializing it. Dangerous and only used by resize
void Strings::reserve(size_t capacity)
{
    string *old = d_str;

    d_str = raw_memory(capacity);
    if (old)
    {
        for (size_t index = 0; 
                    index != capacity and index != d_capacity; 
                    ++index)
            // move for fairness in comparison later, alternatively, copy and 
            // call the destructor on the original, but this is more efficient
            new(d_str + index) string(move(old[index]));
        operator delete(old);
    }
}

\end{minted}
\subsection*{/strings/resize.cpp}
\begin{minted}[frame=lines, linenos, fontsize=\large]{c++}
#include "strings.ih"


// reserves capacity, initializes it, then updates d_capacity
void Strings::resize(size_t capacity)
{
    reserve(capacity);
    for (size_t index = d_capacity; index < capacity; ++index)
        new(d_str + index) string;
    d_capacity = capacity;  
}
\end{minted}
\subsection*{/strings/size.cpp}
\begin{minted}[frame=lines, linenos, fontsize=\large]{c++}
#include "strings.ih"

size_t Strings::size() const
{
    return d_size;   
}
\end{minted}
\subsection*{/strings/strings1.cpp}
\begin{minted}[frame=lines, linenos, fontsize=\large]{c++}
#include "strings.ih"

Strings::Strings()
{
    d_capacity = 1;
    d_size = 0;
    reserve(1);
}

\end{minted}
\subsection*{/strings/strings2.cpp}
\begin{minted}[frame=lines, linenos, fontsize=\large]{c++}
#include "strings.ih"

Strings::Strings(istream &in)
{
    string line;
    while (getline(in, line))
        add(line);
}

\end{minted}
\subsection*{/strings/strings3.cpp}
\begin{minted}[frame=lines, linenos, fontsize=\large]{c++}
#include "strings.ih"

Strings::Strings(char **env)
{
    int amount = 0;
    while (*(env + amount))
        ++amount;
    init(amount, env);
}
\end{minted}
\subsection*{/strings/strings4.cpp}
\begin{minted}[frame=lines, linenos, fontsize=\large]{c++}
#include "strings.ih"

Strings::Strings(int argc, char *argv[])
{
    init(argc, argv);
}
\end{minted}

\subsection*{/stringsdata/stringsdata.ih}
\begin{minted}[frame=lines, linenos, fontsize=\large]{c++}
#include "stringsdata.h"
#include <iostream>

using namespace std;

\end{minted}
\subsection*{/stringsdata/stringsdata.h}
\begin{minted}[frame=lines, linenos, fontsize=\large]{c++}
#ifndef STRINGSDATA_H
#define STRINGSDATA_H

#include <cstddef>
#include <string>

struct StringsData 
{
    size_t size;
    std::string *ptr;

    StringsData(size_t size, std::string *ptr);
};

#endif // STRINGSDATA_H

\end{minted}
\subsection*{/stringsdata/stringsdata1.cpp}
\begin{minted}[frame=lines, linenos, fontsize=\large]{c++}
#include "stringsdata.ih"

StringsData::StringsData(size_t size, string *ptr)
    :
        size(size),
        ptr(ptr)
{}

\end{minted}


\section*{Exercise 56}
[luuk@arch_laptop 56]$ time ./program_original 

real    6m3.744s
user    5m56.094s
sys     0m7.340s
[luuk@arch_laptop 56]$ time ./program_double_ptr 

real    0m0.621s
user    0m0.577s
sys     0m0.043s
[luuk@arch_laptop 56]$ time ./program_placement 

real    0m0.352s
user    0m0.312s
sys     0m0.040s


\section*{Exercise 57}
\subsection*{/cpu/cpu.h}
\begin{minted}[frame=lines, linenos, fontsize=\large]{c++}
#ifndef INCLUDED_CPU_
#define INCLUDED_CPU_

#include "../tokenizer/tokenizer.h"     // the Tokenizer is a component of the
                                        // CPU.

#include "../memory/memory.h"

class CPU
{
    enum 
    {
        NREGISTERS = 5,             // a..e at indices 0..4, respectively
        LAST_REGISTER = NREGISTERS - 1
    };

    struct Operand
    {
        OperandType type;
        int value;
    };
        
    Memory &d_memory;
    Tokenizer d_tokenizer;

    int d_register[NREGISTERS];

public:
    CPU(Memory &memory);
    void run();

private:
    bool error();           // show 'syntax error', and prepare for the 
                            // next input line

    bool execute(Opcode opcode);    // perform the action matching opcode

                            // return a value or a register's or 
                            // memory location's value
    int dereference(Operand const &value);

    bool rvalue(Operand &lhs);  // retrieve an rvalue operand
    bool lvalue(Operand &lhs);  // retrieve an lvalue operand

                            // determine 2 operands, lhs must be an lvalue
    bool operands(Operand &lhs, Operand &rhs);

    bool twoOperands(Operand &lhs, int &lhsValue, int &rhsValue);

                            // store a value in register or memory
    void store(Operand const &lhs, int value);

    void mov();             // assign a value
    void add();             // add values
    void sub();             // subtract values
    void mul();             // multiply values
    void div();             // divide values (remainder: last reg.)
                            // div a b computes a /= b, last reg: %
    void neg();             // negate a value
    void dsp();             // display a value

    int value_fn(Operand const &val);
    int register_fn(Operand const &val);
    int memory_fn(Operand const &val);

    void register_store_fn(Operand const &lhs, int value);
    void memory_store_fn(Operand const &lhs, int value);
};
        
#endif
\end{minted}
\subsection*{/cpu/dereference.cpp}
\begin{minted}[frame=lines, linenos, fontsize=\large]{c++}
#include "cpu.ih"
#include "../enums/enums.h"

int CPU::value_fn(Operand const &val)
{
    return val.value;
}

int CPU::register_fn(Operand const &val)
{
    return d_register[val.value];
}

int CPU::memory_fn(Operand const &val)
{
    return d_memory.load(val.value);
}

int CPU::dereference(Operand const &value)
{
    int (CPU::*fn_list[3])(Operand const &op) = 
        {&CPU::value_fn, &CPU::register_fn, &CPU::memory_fn};
    return (this->*fn_list[static_cast<size_t>(value.type)])(value);
}

\end{minted}
\subsection*{/cpu/execute.cpp}
\begin{minted}[frame=lines, linenos, fontsize=\large]{c++}
#include "cpu.ih"

bool CPU::execute(Opcode opcode)
{
    if (opcode == Opcode::STOP)
        return false;
    if (opcode == Opcode::ERR)
        return error();
    static void (CPU::*fn_list[7])() = {&CPU::mov, &CPU::add, &CPU::sub, 
                                        &CPU::mul, &CPU::div, &CPU::neg, 
                                        &CPU::dsp};
    (this->*fn_list[static_cast<size_t>(opcode)])();
    return true;                // continue
}
\end{minted}
\subsection*{/cpu/store.cpp}
\begin{minted}[frame=lines, linenos, fontsize=\large]{c++}
#include "cpu.ih"

void CPU::register_store_fn(Operand const &lhs, int value)
{
    d_register[lhs.value] = value;
}

void CPU::memory_store_fn(Operand const &lhs, int value)
{
    d_memory.store(lhs.value, value);
}

void CPU::store(Operand const &lhs, int value)
{
    static void (CPU::*fn_list[2])(Operand const &lhs, int value) = 
        {&CPU::register_store_fn, &CPU::memory_store_fn};
    (this->*fn_list[static_cast<size_t>(lhs.type)])(lhs, value);
}   
\end{minted}



\end{document}
