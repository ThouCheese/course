\documentclass{article}
\usepackage[utf8]{inputenc}
\usepackage[T1]{fontenc}
\usepackage[utf8]{inputenc}
\usepackage{lmodern}
\usepackage[a4paper, margin=1in]{geometry}

\usepackage{minted}
\large
\title{C++ Assignment 3}
\begin{document}
\begin{titlepage}
    \begin{center}
        \line(1,0){300}\\
        [0.65cm]
        \huge{\bfseries Assignment III}\\
        \line(1,0){300}\\
        \textsc{\Large C++ Course Part I}\\
        \textsc{\LARGE \today}\\
        [5.5cm]     
    \end{center}
    \begin{flushright}
        \textsc{\Large L. Wester\\S2755351}\\
        [0.5cm]
    \end{flushright}
\end{titlepage}
\section*{Exercise 20}
\subsection*{/find\_period.cpp}
\begin{minted}[frame=lines, linenos, fontsize=\large]{c++}
#include "main.ih"

bool contains_period(size_t argc, char *argv[])
{
    for (size_t index = 0; index != argc; ++index)
        if (std::string{argv[index]}.find('.') != std::string::npos)
            return true;
    return false;
}
\end{minted}
\subsection*{/main.cpp}
\begin{minted}[frame=lines, linenos, fontsize=\large]{c++}
#include "main.ih"

int main(int argc, char *argv[])
{
    --argc;
    ++argv;
    if (contains_period(argc, argv))
        cout << sum(argc, argv, 0.0) << '\n';
    else
        cout << sum(argc, argv, 0) << '\n';
}

\end{minted}
\subsection*{/main.ih}
\begin{minted}[frame=lines, linenos, fontsize=\large]{c++}
#include <iostream>

using namespace std;

bool contains_period(size_t argc, char *argv[]);

double sum(int argc, char *argv[], double result);

int sum(int argc, char *argv[], int result);

\end{minted}
\subsection*{/sum1.cpp}
\begin{minted}[frame=lines, linenos, fontsize=\large]{c++}
#include "main.ih"

double sum(int argc, char *argv[], double result)
{
    for (size_t index = 0; index != argc; ++index)
        result += stod(argv[index]);
    return result;
}

\end{minted}
\subsection*{/sum2.cpp}
\begin{minted}[frame=lines, linenos, fontsize=\large]{c++}
#include "main.ih"

int sum(int argc, char *argv[], int result)
{
    for (size_t index = 0; index != argc; ++index)
        result += stoi(argv[index]);
    return result;
}

\end{minted}

\section*{Exercise 21}
\subsection*{/bound\_call.cpp}
\begin{minted}[frame=lines, linenos, fontsize=\large]{c++}
#include "main.ih"

void boundCall(int argc, char const *argv[])
{
    print_struct(combine(argc, argv));
}

\end{minted}
\subsection*{/combine.cpp}
\begin{minted}[frame=lines, linenos, fontsize=\large]{c++}
#include "main.ih"

ReturnValues combine(int argc, char const *argv[])
{
    ReturnValues result{false, 0, ""};
    if (argc == 1)
        return result;
    result.nr = std::stoul(argv[1]);
    if (argc < result.nr or result.nr == 0)
        return result;
    result.value = argv[result.nr - 1];
    result.ok = true;
    return result;
}
\end{minted}
\subsection*{/main.cpp}
\begin{minted}[frame=lines, linenos, fontsize=\large]{c++}
#include "main.ih"

int main(int argc, char const *argv[])
{
    if (not structCall(argc, argv))
        return 1;
    boundCall(argc, argv);
}

\end{minted}
\subsection*{/main.ih}
\begin{minted}[frame=lines, linenos, fontsize=\large]{c++}
#include <iostream>

using namespace std;

struct ReturnValues
{
    bool ok;
    size_t nr;
    string value;
};

void print_struct(ReturnValues const &values);

bool structCall(int argc, char const *argv[]);

void boundCall(int argc, char const *argv[]);

ReturnValues combine(int argc, char const *argv[]);

\end{minted}
\subsection*{/print\_struct.cpp}
\begin{minted}[frame=lines, linenos, fontsize=\large]{c++}
#include "main.ih"

void print_struct(ReturnValues const &values)
{
    cout << "Value at " << values.nr << " is " << values.value << '\n';
}
\end{minted}
\subsection*{/struct\_call.cpp}
\begin{minted}[frame=lines, linenos, fontsize=\large]{c++}
#include "main.ih"

bool structCall(int argc, char const *argv[])
{
    ReturnValues local = combine(argc, argv);
    if (not local.ok)
    {
        cerr << "Usage: program <number> [arguments]\n";
        return false;
    }
    print_struct(local);
    return true;
}

\end{minted}

\section*{Exercise 22}
\subsection*{/count\_chars.cpp}
\begin{minted}[frame=lines, linenos, fontsize=\large]{c++}
#include "main.ih"

size_t count_chars()
{
    size_t char_count = 0;
    string line;
    while (getline(std::cin, line)) 
        char_count += line.length() + !cin.eof();
    return char_count;
}
\end{minted}
\subsection*{/count\_lines.cpp}
\begin{minted}[frame=lines, linenos, fontsize=\large]{c++}
#include "main.ih"

size_t count_lines()
{
    size_t line_count = 0;
    string line;
    while (getline(std::cin, line)) 
        line_count += 1;
    return line_count;
}

\end{minted}
\subsection*{/count\_words.cpp}
\begin{minted}[frame=lines, linenos, fontsize=\large]{c++}
#include "main.ih"

size_t count_words()
{
    size_t word_count = 0;
    string line;
    while (getline(std::cin, line)) 
    {
        size_t words_in_string = 0;
        bool was_space = true;
        for (char ch : line)
            if (ch == ' ')
            {
                if (not was_space)
                    words_in_string += 1;
            }
            else
                was_space = false;
        word_count += words_in_string + not was_space;
    }
    return word_count;
}
    
\end{minted}
\subsection*{/err.cpp}
\begin{minted}[frame=lines, linenos, fontsize=\large]{c++}
#include "main.ih"

bool err(int const argc, char const *argv[])
{
    if (argc != 2)
        cerr << "Usage: Wc [-c, -w, -l]\n";
    else if (argv[1][0] != '-')
        cerr << "Expected '-'\n";
    else
        return false;
    return true;
}

\end{minted}
\subsection*{/main.cpp}
\begin{minted}[frame=lines, linenos, fontsize=\large]{c++}
#include "main.ih"

int main(int const argc, char const *argv[])
{
    if (err(argc, argv))
        return 1;
    size_t count = 0;
    switch (argv[1][1]) {
        case 'c':
            count = count_chars();
            break;
        case 'w':
            count = count_words();
            break;
        case 'l':
            count = count_lines();
            break;
        default:
            cerr << "Unknown option " << argv[1][1] << '\n';
            return 1;
    }
    cout << count << '\n';
}

\end{minted}
\subsection*{/main.ih}
\begin{minted}[frame=lines, linenos, fontsize=\large]{c++}
#include <iostream>

using namespace std;

bool err(int const argc, char const *argv[]);

size_t count_chars();
size_t count_words();
size_t count_lines();

\end{minted}

\section*{Exercise 23}
\subsection*{/get\_env\_len.cpp}
\begin{minted}[frame=lines, linenos, fontsize=\large]{c++}
#include "main.ih"

size_t get_env_len()
{
    size_t length = 0;
    for (char **env = environ; *env; ++env)
        ++length;
    return length;
}
\end{minted}
\subsection*{/main.cpp}
\begin{minted}[frame=lines, linenos, fontsize=\large]{c++}
#include "main.ih"

int main()
{
    size_t env_length = get_env_len();
    string list[env_length];
    for (size_t index = 0; index != env_length; ++index)
        list[index] = string{environ[index]};

    quicksort(list, 0, env_length - 1);

    for (size_t index = 0; index != env_length; ++index)
        cout << list[index] << '\n';
}

\end{minted}
\subsection*{/main.ih}
\begin{minted}[frame=lines, linenos, fontsize=\large]{c++}
#include <iostream>

using namespace std;

extern char **environ;


size_t get_env_len();
string *get_env_list(size_t const length, string *list);

void quicksort(string list[], size_t left, size_t right);
size_t partition(string list[], size_t left, size_t right);

\end{minted}
\subsection*{/partition.cpp}
\begin{minted}[frame=lines, linenos, fontsize=\large]{c++}
#include "main.ih"

size_t partition(string list[], size_t left, size_t right)
{
    string pivot = list[left];
    size_t pivot_new = left;
    for (size_t index = left + 1; index != right + 1; ++index)
    {
        size_t char_pos = 0;
        while (tolower(list[index][char_pos]) == tolower(pivot[char_pos]))
            ++char_pos;

        if (list[index][char_pos] < pivot[char_pos])
            swap(list[++pivot_new], list[index]);
    }
    swap(list[pivot_new], list[left]);
    return pivot_new;
}
\end{minted}
\subsection*{/quicksort.cpp}
\begin{minted}[frame=lines, linenos, fontsize=\large]{c++}
#include "main.ih"

void quicksort(string list[], size_t left, size_t right)
{
    if (left < right)
    {
        size_t mid = partition(list, left, right);
        quicksort(list, left, mid - 1);
        quicksort(list, mid + 1, right);
    }
}
\end{minted}
\subsection*{/swap.cpp}
\begin{minted}[frame=lines, linenos, fontsize=\large]{c++}
#include "main.ih"

void swap(string *string1, string *string2)
{
    string *temp = string1;
    string1 = string2;
    string2 = temp;
}
\end{minted}

\section*{Exercise 25}
\subsection*{/insert.cpp}
\begin{minted}[frame=lines, linenos, fontsize=\large]{c++}
#include "main.ih"

void insert(Subject *subjects, 
            Subject subject,
            size_t position, 
            size_t length)
{
    for (size_t index = length - 1; index != position;)
    {
        --index;
        subjects[index + 1] = subjects[index];
    }
    subjects[position] = subject;
}
\end{minted}
\subsection*{/main.cpp}
\begin{minted}[frame=lines, linenos, fontsize=\large]{c++}
#include "main.ih"

int main(int argc, char const *argv[])
{
    if (argc != 3)
    {
        cerr << "Usage: program n_people seed";
        return 1;
    }
    size_t const n_people = std::stoul(argv[1]);
    size_t const seed = std::stoul(argv[2]);
    srandom(seed);
    Subject *subjects = new Subject[n_people];
    populate_sorted(n_people, subjects);
    print_subjects(subjects, n_people);
}

\end{minted}
\subsection*{/main.ih}
\begin{minted}[frame=lines, linenos, fontsize=\large]{c++}
#include <iostream>
#include <stdlib.h>
#include <vector>

using namespace std;

struct Subject
{
    bool brown;
    size_t order;
};

void populate_sorted(size_t const n_people, 
                     Subject *subjects);

void print_subjects(Subject const *subjects, size_t const size);

void insert(Subject *subjects, 
            Subject subject,
            size_t position, 
            size_t length);

\end{minted}
\subsection*{/populate\_sorted.cpp}
\begin{minted}[frame=lines, linenos, fontsize=\large]{c++}
#include "main.ih"

void populate_sorted(size_t const n_people,
                     Subject *subjects)
{
    for (size_t n_person = 0; n_person != n_people; ++n_person)
    {
        size_t index = 0;
        for (; index != n_person; ++index)
            if (not subjects[index].brown)
                break;
        // cout << index << '\n';
        insert(subjects, Subject{random() % 2, n_person}, index, n_people);
    }
}

\end{minted}
\subsection*{/print\_subjects.cpp}
\begin{minted}[frame=lines, linenos, fontsize=\large]{c++}
#include "main.ih"

void print_subjects(Subject const *subjects, size_t const size)
{
    for (size_t index = 0; index != size; ++index)
        std::cout 
            << (index + 1) << ": "
            << (subjects[index].brown ? "brown (" : "blue (")
            << (subjects[index].order) << ")\n";
}
\end{minted}

\section*{Exercise 27}
\subsection*{/combis.cpp}
\begin{minted}[frame=lines, linenos, fontsize=\large]{c++}
#include "main.ih"

void combis(size_t idx, bool bits[])
{
    size_t cur_total = 0;
    for (size_t index = 0; index != idx; ++index)
        cur_total += bits[index];
    for (size_t index = idx; index != nTotal; ++index)
    {
        if (cur_total + (nTotal - index) < nRequired)
            break;
        bits[index] = true;
        combis(index + 1, bits);
        bits[index] = false;
    }
    if (cur_total >= nRequired)
        print(bits);
}

\end{minted}
\subsection*{/data.cpp}
\begin{minted}[frame=lines, linenos, fontsize=\large]{c++}
#include "main.ih"

size_t nTotal;
size_t nRequired;

\end{minted}
\subsection*{/main.cpp}
\begin{minted}[frame=lines, linenos, fontsize=\large]{c++}
#include "main.ih"

int main(int argc, char const *argv[])
{
    if (argc != 3)
    {
        cerr << "Usage: program nTotal nRequired\n";
        return 1;
    }
    nTotal = std::stoul(argv[1]);
    nRequired = std::stoul(argv[2]);
    if (nTotal < nRequired) 
    {
        cerr << "nTotal must be greater than nRequired\n";
        return 1;
    }
    bool bits[nTotal];
    combis(0, bits);
}

\end{minted}
\subsection*{/main.ih}
\begin{minted}[frame=lines, linenos, fontsize=\large]{c++}
#include <iostream>

using namespace std;

extern size_t nTotal;
extern size_t nRequired;

void combis(size_t index, bool bits[]);
void print(bool bits[]);

\end{minted}
\subsection*{/print.cpp}
\begin{minted}[frame=lines, linenos, fontsize=\large]{c++}
#include "main.ih"

void print(bool bits[])
{
    for (size_t index = 0; index != nTotal; ++index)
        if (bits[index])
            std::cout << index;
    std::cout << '\n';
}

\end{minted}

\section*{Exercise 28}
\subsection*{/estimate\_b.cpp}
\begin{minted}[frame=lines, linenos, fontsize=\large]{c++}
// expecting the value to compute (lhs + rhs) for, and the a-term (i.e., square root estimate)
// of all but the least significant pair of digits of that value.

#include "main.ih"

size_t estimateB(size_t pq, size_t lhs)
{
    return (pq - 100 * lhs * lhs) / (20 * lhs);
}

\end{minted}
\subsection*{/lookup\_sqrt.cpp}
\begin{minted}[frame=lines, linenos, fontsize=\large]{c++}
#include "main.ih"

size_t lookupSqrt(size_t num)
{
    for (size_t index = 0; index != 10; ++index)
        if (index * index > num)
            return index - 1;

    // // Alternative method using switch:

    // switch(num)
    // {
    //       default:
    //           return 0;
    //     case 1 ... 3:
    //         return 1;
    //     case 4 ... 8:
    //         return 2;
    //     case 9 ... 15:
    //         return 3;
    //     case 16 ... 24:
    //         return 4;
    //     case 25 ... 35:
    //         return 5;
    //     case 36 ... 48:
    //         return 6;
    //     case 49 ... 63:
    //         return 7;
    //     case 64 ... 80:
    //         return 8;
    //     case 81 ... 99:
    //         return 9;
    // }
}

\end{minted}
\subsection*{/main.cpp}
\begin{minted}[frame=lines, linenos, fontsize=\large]{c++}
#include "main.ih"

int main()
{
    while(true)
    {
        cout << "? ";
        string text;
        if (not getline(cin, text))
            break;
        size_t input = stoul(text);
        size_t sqrt = sqrtGn(input);
        cout << "sqrt: " << sqrt << " (" << sqrt * sqrt << ")\n";
    }
}
\end{minted}
\subsection*{/main.ih}
\begin{minted}[frame=lines, linenos, fontsize=\large]{c++}
#include <iostream>
#include <string>

using namespace std;

size_t sqrtGn(size_t input);
size_t lookupSqrt(size_t num);
size_t estimateB(size_t pq, size_t lhs);
size_t nextSqrt(size_t pq, size_t lhs);
\end{minted}
\subsection*{/next\_sqrt.cpp}
\begin{minted}[frame=lines, linenos, fontsize=\large]{c++}
#include "main.ih"

size_t nextSqrt(size_t pq, size_t lhs)
{
    size_t best = estimateB(pq, lhs);
    size_t sqrt = 10 * lhs + best;
    while (sqrt * sqrt > pq)
        sqrt = 10 * lhs + --best;
    return sqrt;
}

\end{minted}
\subsection*{/sqrt\_gn.cpp}
\begin{minted}[frame=lines, linenos, fontsize=\large]{c++}
#include "main.ih"

size_t sqrtGn(size_t input)
{
    if (input / 100 == 0)
        return lookupSqrt(input);
    else
    {
        size_t test = sqrtGn(input / 100);
        size_t a = nextSqrt(input, test);
        return a;
    }
}
\end{minted}

\section*{Exercise 30}
\subsection*{/call\_ref.cpp}
\begin{minted}[frame=lines, linenos, fontsize=\large]{c++}
#include "main.h"

void callRef(string const &prog)
{
    for (size_t idx = 0; idx < 10000000; ++idx)
        fun2(prog);
}
\end{minted}
\subsection*{/call\_value.cpp}
\begin{minted}[frame=lines, linenos, fontsize=\large]{c++}
#include "main.h"

void callValue(string const &prog)
{
    for (size_t idx = 0; idx < 10000000; ++idx)
        fun(prog);
}
\end{minted}
\subsection*{/fun1.cpp}
\begin{minted}[frame=lines, linenos, fontsize=\large]{c++}
#include "main.h"

void fun(string s)
{
    size_t sum = 0;
    for (size_t idx = 0; idx < s.length(); ++idx)
        sum += s[idx];
}

\end{minted}
\subsection*{/fun2.cpp}
\begin{minted}[frame=lines, linenos, fontsize=\large]{c++}
#include "main.h"

void fun2(string const &s)
{
    size_t sum = 0;
    for (size_t idx = 0; idx < s.length(); ++idx)
        sum += s[idx];
}

\end{minted}
\subsection*{/main.cpp}
\begin{minted}[frame=lines, linenos, fontsize=\large]{c++}
#include "main.h"

int main(int argc, char *argv[])
{
    callValue(argv[0]);
    callRef(argv[0]);
}
\end{minted}
\subsection*{/main.h}
\begin{minted}[frame=lines, linenos, fontsize=\large]{c++}
#include <string>
#include <iostream>

using namespace std;

void fun(string s);
void fun2(string const &s);
void callValue(string const &prog);
void callRef(string const &prog);

\end{minted}


\end{document}
