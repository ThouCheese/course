\documentclass{article}
\usepackage[utf8]{inputenc}
\usepackage[T1]{fontenc}
\usepackage[utf8]{inputenc}
\usepackage{lmodern}
\usepackage[a4paper, margin=1in]{geometry}

\usepackage{minted}
\large
\title{C++ Assignment 3}
\begin{document}
\begin{titlepage}
    \begin{center}
        \line(1,0){300}\\
        [0.65cm]
        \huge{\bfseries Assignment III}\\
        \line(1,0){300}\\
        \textsc{\Large C++ Course Part I}\\
        \textsc{\LARGE \today}\\
        [5.5cm]     
    \end{center}
    \begin{flushright}
        \textsc{\Large L. Wester\\S2755351}\\
        [0.5cm]
    \end{flushright}
\end{titlepage}
\section*{Exercise 40}
1. The main difference is that an array cannot be incremented the way a pointer 
can to get the next element. Arrays are immutable, pointers are mutable.

2. References cannot be re-assigned, point to NULL, be explicitly dereferenced 
or be incremented to the next value in memory, whereas pointers can do all these
things.

3. 









4. The pointer arithmetic is a small subset of arithmetic operations that are allowed on numbers. It is restricted to ++, --, + and -. These operations are allowed on pointers. It also concerns itself with which type of operands are allowed and whether these operations return a number or another pointer.

5. It is preffered to access an element using a pointer as opposed to an index expression because pointers are generally faster.

\subsection*{/main.cpp}
\begin{minted}[frame=lines, linenos, fontsize=\large]{c++}
#include <iostream>

int main()
{
    char *ptr = "hoi\n";
    char arr[] = "hoi\n";
    ptr[1] = '0';
    arr[1] = '0';
    ++arr;
    std::cout << *ptr << '\n';
    std::cout << *arr << '\n';
}
\end{minted}

\section*{Exercise 41}
\subsection*{/main.cpp}
\begin{minted}[frame=lines, linenos, fontsize=\large]{c++}
#include <iostream>

extern char **environ;

int main(int argc, char *argv[])
{
    int offset = environ - argv;

    for (char **arg_ptr = argv; *arg_ptr and *(arg_ptr + offset); ++arg_ptr)
        std::cout << *(arg_ptr + offset) << '\n';

    for (char **env_ptr = environ; *env_ptr and *(env_ptr - offset); ++env_ptr)
        std::cout << *(env_ptr - offset) << '\n';
}

\end{minted}

\section*{Exercise 42}
\subsection*{/main.ih}
\begin{minted}[frame=lines, linenos, fontsize=\large]{c++}
#include "charcount/charcount.h"
#include <iostream>

using namespace std;

void showChar(char ch);
\end{minted}
\subsection*{/main.cpp}
\begin{minted}[frame=lines, linenos, fontsize=\large]{c++}
#include "main.ih"

int main()
{
    CharCount counter;
    counter.count(std::cin);
    for (size_t index = 0; index != counter.info().nChar; ++index)
    {
        showChar(counter.info().ptr[index].ch);
        cout << " found " << counter.info().ptr[index].count << " times\n";
    }
}
\end{minted}
\subsection*{/show\_char.cpp}
\begin{minted}[frame=lines, linenos, fontsize=\large]{c++}
#include "main.ih"

void showChar(char ch)
{
    cout << "char \'";
    switch (ch) {
        case '\t':
            cout << "\\t";
        break;
        case '\n':
            cout << "\\n";
        break;
        default:
            cout << ch;
    }
    cout << '\'';
}
\end{minted}
\subsection*{/char/char.ih}
\begin{minted}[frame=lines, linenos, fontsize=\large]{c++}
#include "char.h"
#include <iostream>

using namespace std;

\end{minted}
\subsection*{/char/char.h}
\begin{minted}[frame=lines, linenos, fontsize=\large]{c++}
#ifndef CHAR_H
#define CHAR_H

#include <cstddef>

struct Char 
{
    char ch;
    size_t count;

    Char() = default;
    Char(char ch, size_t count);
};

#endif // CHAR_H

\end{minted}
\subsection*{/char/char1.cpp}
\begin{minted}[frame=lines, linenos, fontsize=\large]{c++}
#include "char.ih"

Char::Char(char ch, size_t count)
    :
        ch(ch),
        count(count)
{}
\end{minted}

\subsection*{/charcount/charcount.ih}
\begin{minted}[frame=lines, linenos, fontsize=\large]{c++}
#include "charcount.h"
#include <iostream>

using namespace std;

\end{minted}
\subsection*{/charcount/charcount.h}
\begin{minted}[frame=lines, linenos, fontsize=\large]{c++}
#ifndef CHARCOUNT_H
#define CHARCOUNT_H

#include "../enums/action.h"
#include "../charinfo/charinfo.h"
#include <iostream>

class CharCount 
{
    size_t d_work_index = 0;
    CharInfo d_char_info;
public:
    void count(std::istream &in);
    CharInfo const &info() const;

private:
    Action locate(char ch);
    void add(char ch);
    void insert(char ch);
    void append(char ch);
};

#endif // CHARCOUNT_H

\end{minted}
\subsection*{/charcount/add.cpp}
\begin{minted}[frame=lines, linenos, fontsize=\large]{c++}
#include "charcount.ih"

void CharCount::add(char ch)
{
    d_char_info.ptr[d_work_index].count++;
}
\end{minted}
\subsection*{/charcount/append.cpp}
\begin{minted}[frame=lines, linenos, fontsize=\large]{c++}
#include "charcount.ih"

void CharCount::append(char ch)
{
    Char *old = d_char_info.ptr;
    if (not old)
    {
        d_char_info.nChar = 1;
        d_char_info.ptr = new Char[1];
        d_char_info.ptr[0] = Char(ch, 1);
        return;
    }
    d_char_info.ptr = new Char[++d_char_info.nChar];
    for (size_t index = 0; index != d_char_info.nChar - 1; ++index)
        d_char_info.ptr[index] = old[index];
    delete[] old;
    d_char_info.ptr[d_char_info.nChar - 1] = Char(ch, 1);
}

\end{minted}
\subsection*{/charcount/count.cpp}
\begin{minted}[frame=lines, linenos, fontsize=\large]{c++}
#include "charcount.ih"

void CharCount::count(std::istream &in)
{
    char ch;
    cin >> noskipws;

    while (cin >> ch)
    {
        Action action = locate(ch);
        if (action == Action::INSERT)
            insert(ch);
        else if (action == Action::APPEND)
            append(ch);
        else if (action == Action::ADD)
            add(ch);
    }
}

\end{minted}
\subsection*{/charcount/info.cpp}
\begin{minted}[frame=lines, linenos, fontsize=\large]{c++}
#include "charcount.ih"

CharInfo const &CharCount::info() const
{
    return d_char_info;
}
\end{minted}
\subsection*{/charcount/insert.cpp}
\begin{minted}[frame=lines, linenos, fontsize=\large]{c++}
#include "charcount.ih"

void CharCount::insert(char ch)
{
    Char *old = d_char_info.ptr;
    d_char_info.ptr = new Char[++d_char_info.nChar];

    size_t index;
    for (index = 0; index != d_work_index; ++index)
        d_char_info.ptr[index] = old[index];
    d_char_info.ptr[d_work_index] = Char(ch, 1);
    while (++index != d_char_info.nChar)
        d_char_info.ptr[index] = old[index - 1];

    delete[] old;
}
\end{minted}
\subsection*{/charcount/locate.cpp}
\begin{minted}[frame=lines, linenos, fontsize=\large]{c++}
#include "charcount.ih"

Action CharCount::locate(char ch)
{
    if (not d_char_info.ptr)
        return Action::APPEND;
    for (size_t index = 0; index != d_char_info.nChar; ++index)
    {
        char cur_char = d_char_info.ptr[index].ch;
        if (cur_char == ch)
        {
            d_work_index = index;
            return Action::ADD;
        }
        else if (cur_char > ch)
        {
            d_work_index = index;
            return Action::INSERT;
        }
    }
    return Action::APPEND;
}
\end{minted}

\subsection*{/charinfo/charinfo.ih}
\begin{minted}[frame=lines, linenos, fontsize=\large]{c++}
#include "charinfo.h"
#include <iostream>

using namespace std;

\end{minted}
\subsection*{/charinfo/charinfo.h}
\begin{minted}[frame=lines, linenos, fontsize=\large]{c++}
#ifndef CHARINFO_H
#define CHARINFO_H

#include "../char/char.h"
#include "../enums/action.h"

struct CharInfo 
{
    size_t nChar = 0;
    Char *ptr = 0;
};

#endif // CHARINFO_H
\end{minted}

\subsection*{/enums/action.h}
\begin{minted}[frame=lines, linenos, fontsize=\large]{c++}
#ifndef ACTION_H
#define ACTION_H

enum Action
{
    INSERT,
    ADD,
    APPEND,
};

#endif // ACTION_H
\end{minted}


\section*{Exercise 43}
----------------------------------------------------------------------------
  definition:         rewrite: 
----------------------------------------------------------------------------
  int x[8];           x[4]            

pointer notation:  *(x + 4)
       semantics:  x + 4 points to the location of the 4th int beyond x.
                   That element is reached using the dereference operator (*)

----------------------------------------------------------------------------
  int x[8];           x[3] = x[2];

pointer notation:  *(x + 3) = *(x + 2);
       semantics:  Copy the int at location x + 3 to the location x + 2
----------------------------------------------------------------------------
  char *argv[8];      cout << argv[2];

pointer notation:  cout << *(argv + 2);
       semantics:  Take the 2nd element beyond x, deference it and print it
----------------------------------------------------------------------------
  int x[8];           &x[10] - &x[3];
 
pointer notation:  &*(x + 10) - &*(x + 3);
       semantics:  take the address of the value at the 10th location beyond
                   x, then subtract the address of the value at the 10th location beyond x.
                   Equivalently: subtract 10 from 3
----------------------------------------------------------------------------
  char *argv[8];      argv[0]++;

pointer notation:  (*argv)++
       semantics:  Take the first char* that argv points to and increment it
                   beyond its first value, effectively leaving out the first
                   char
----------------------------------------------------------------------------
  char *argv[8];      argv++[0];

pointer notation:  *(argv++)
       semantics:  Take the first char string that argv points to, then 
                   increment argv beyond that string to leave it out
----------------------------------------------------------------------------
  char *argv[8];      ++argv[0];

pointer notation:  *++argv
       semantics:  First increment argv to forget the first string it
                   contains, then take the now first previously second 
                   string
----------------------------------------------------------------------------
  char **argv;        ++argv[0][2];

pointer notation:  *(*(++argv) + 2)
       semantics:  Increment argv to forget about its first string, then 
                   take now fist, previously second string, then, from that
                   string take the third character 
----------------------------------------------------------------------------
\subsection*{/test.cpp}
\begin{minted}[frame=lines, linenos, fontsize=\large]{c++}
#include <iostream>

int main()
{
    char *argv[8] = {"12345678", 
                     "hoi doeg",
                     "hallo ga"};
    argv++[0];
    std::cout << *(++argv) << '\n';
}
\end{minted}

\section*{Exercise 44}
\subsection*{/main.ih}
\begin{minted}[frame=lines, linenos, fontsize=\large]{c++}
#include <iostream>

using namespace std;

void inv_identity(int row[][10]);

\end{minted}
\subsection*{/inv\_identity.cpp}
\begin{minted}[frame=lines, linenos, fontsize=\large]{c++}
#include "main.ih"

void inv_identity(int row[][10])
{
    for (int (*rowPtr)[10] = row; rowPtr != row + 10; ++rowPtr)
    {
        for (int *colPtr = *rowPtr, *mid = *rowPtr + (rowPtr - row); 
             colPtr != *rowPtr + 10; 
             ++colPtr)
        {
            *colPtr = colPtr == mid ? 0 : 1;
        }
    }
}
\end{minted}
\subsection*{/main.cpp}
\begin{minted}[frame=lines, linenos, fontsize=\large]{c++}
#include "main.ih"

int main()
{
    int square[10][10];
    int (*row)[10] = square;
    inv_identity(square);
    for (size_t irow = 0; irow != 10; ++irow)
    {
        for (size_t icolumn = 0; icolumn != 10; ++icolumn)
            cout << square[irow][icolumn] << ' ';
        cout << '\n';
    }
}
\end{minted}

\section*{Exercise 45-46}
\subsection*{/main.ih}
\begin{minted}[frame=lines, linenos, fontsize=\large]{c++}
#include "strings/strings.h"
#include "filter/filter.h"
#include <iostream>

extern char **environ;

using namespace std;

\end{minted}
\subsection*{/main.cpp}
\begin{minted}[frame=lines, linenos, fontsize=\large]{c++}
#include "main.ih"

int main()
{
    Strings str(environ);
    Filter filter(str);
    filter.display();
}
\end{minted}
\subsection*{/filter/filter.ih}
\begin{minted}[frame=lines, linenos, fontsize=\large]{c++}
#include "filter.h"
#include <iostream>

using namespace std;

\end{minted}
\subsection*{/filter/filter.h}
\begin{minted}[frame=lines, linenos, fontsize=\large]{c++}
#ifndef FILTER_H
#define FILTER_H

#include "../strings/strings.h"

class Filter 
{
    Strings const &d_strings; // store a reference because a value is woefully inefficient
    size_t d_index = 0;
public:
    Filter(Strings const &strings);
    void display();
private:
    bool getline(std::string &line);
};

#endif // FILTER_H

\end{minted}
\subsection*{/filter/display.cpp}
\begin{minted}[frame=lines, linenos, fontsize=\large]{c++}
#include "filter.ih"

void Filter::display()
{
    size_t line_num = 0;
    string line;
    while (true)
    {
        if (not this->getline(line))
            return; // all lines were empty
        if (line.find_first_not_of(" \t") != string::npos) // non empty
            break;
    }
    cout << line << '\n';

    size_t n_empty = 0; // number of empty lines
    while (this->getline(line))
        if (line.find_first_not_of(" \t") == string::npos)
            ++n_empty;
        else
        {
            cout << string(n_empty, '\n') << line << '\n';
            n_empty = 0;
        }
}

\end{minted}
\subsection*{/filter/filter1.cpp}
\begin{minted}[frame=lines, linenos, fontsize=\large]{c++}
#include "filter.ih"

Filter::Filter(Strings const &strings)
    :
        d_strings(strings)
{}

\end{minted}
\subsection*{/filter/getline.cpp}
\begin{minted}[frame=lines, linenos, fontsize=\large]{c++}
#include "filter.ih"

bool Filter::getline(string &line)
{
    if (d_index == d_strings.size())
        return false;
    line = d_strings.at(d_index++);
    return true;
}
\end{minted}

\subsection*{/strings/strings.ih}
\begin{minted}[frame=lines, linenos, fontsize=\large]{c++}
#include "strings.h"
#include <iostream>

using namespace std;

\end{minted}
\subsection*{/strings/strings.h}
\begin{minted}[frame=lines, linenos, fontsize=\large]{c++}
#ifndef STRINGS_H
#define STRINGS_H

#include "../stringsdata/stringsdata.h"
#include <string>
#include <istream>

class Strings 
{
    size_t d_size;                     // number of stored strings
    std::string *d_str;                // pointer to `d_str` strings
    std::string d_empty;
public:
    Strings();
    Strings(std::istream &in);
    Strings(char **env);
    Strings(int argc, char *argv[]);
    size_t size() const;
    std::string *data();
    std::string const &at(size_t index) const;
    std::string &at(size_t index);
    StringsData release(); // danger! leaks!
    void add(std::string the_next_string);

private:
    std::string &at_backdoor(size_t index) const;
    void realloc(size_t new_size);
};

#endif // STRINGS_H

\end{minted}
\subsection*{/strings/add.cpp}
\begin{minted}[frame=lines, linenos, fontsize=\large]{c++}
#include "strings.ih"

void Strings::add(string the_next_string)
{
    realloc(d_size + 1);
    d_str[d_size - 1] = the_next_string;
}
\end{minted}
\subsection*{/strings/at.cpp}
\begin{minted}[frame=lines, linenos, fontsize=\large]{c++}
#include "strings.ih"

string &Strings::at(size_t index)
{
    return at_backdoor(index);
}
\end{minted}
\subsection*{/strings/at\_backdoor.cpp}
\begin{minted}[frame=lines, linenos, fontsize=\large]{c++}
#include "strings.ih"

string &Strings::at_backdoor(size_t index) const
{
    // return index < d_size ? &d_str[index] : &d_empty;
    static string empty;
    return index < d_size ? d_str[index] : empty; 
}

\end{minted}
\subsection*{/strings/at\_const.cpp}
\begin{minted}[frame=lines, linenos, fontsize=\large]{c++}
#include "strings.ih"

string const &Strings::at(size_t index) const
{
    return at_backdoor(index);
}
\end{minted}
\subsection*{/strings/data.cpp}
\begin{minted}[frame=lines, linenos, fontsize=\large]{c++}
#include "strings.ih"

string *Strings::data()
{
    return d_str;
}
\end{minted}
\subsection*{/strings/realloc.cpp}
\begin{minted}[frame=lines, linenos, fontsize=\large]{c++}
#include "strings.ih"

void Strings::realloc(size_t new_size)
{
    string *old = d_str;
    d_str = new string[new_size];
    if (old)
        for (size_t index = 0; index != d_size and index != new_size; ++index)
            d_str[index] = old[index];
    d_size = new_size;
    if (old)
        delete[] old;
}
\end{minted}
\subsection*{/strings/release.cpp}
\begin{minted}[frame=lines, linenos, fontsize=\large]{c++}
#include "strings.ih"

StringsData Strings::release()
{
    StringsData data(d_size, d_str);
    Strings();
    return data;
}
\end{minted}
\subsection*{/strings/size.cpp}
\begin{minted}[frame=lines, linenos, fontsize=\large]{c++}
#include "strings.ih"

size_t Strings::size() const
{
    return d_size;   
}
\end{minted}
\subsection*{/strings/strings1.cpp}
\begin{minted}[frame=lines, linenos, fontsize=\large]{c++}
#include "strings.ih"

Strings::Strings()
    :
        d_size(0),
        d_str(0)
{}

\end{minted}
\subsection*{/strings/strings2.cpp}
\begin{minted}[frame=lines, linenos, fontsize=\large]{c++}
#include "strings.ih"

Strings::Strings(istream &in)
    :
        Strings()
{
    string line;
    while (getline(in, line))
    {
        realloc(d_size + 1);
        d_str[d_size - 1] = line;
    }
}
\end{minted}
\subsection*{/strings/strings3.cpp}
\begin{minted}[frame=lines, linenos, fontsize=\large]{c++}
#include "strings.ih"

Strings::Strings(char **env)
    :
        Strings()
{
    while (*env)
        add(*(env++));
}
\end{minted}
\subsection*{/strings/strings4.cpp}
\begin{minted}[frame=lines, linenos, fontsize=\large]{c++}
#include "strings.ih"

Strings::Strings(int argc, char *argv[])
    :
        d_size(argc),
        d_str(new string[argc])
{
    for (size_t index = 0; index != static_cast<size_t>(argc); ++index)
        d_str[index] = argv[index];
}
\end{minted}

\subsection*{/stringsdata/stringsdata.ih}
\begin{minted}[frame=lines, linenos, fontsize=\large]{c++}
#include "stringsdata.h"
#include <iostream>

using namespace std;

\end{minted}
\subsection*{/stringsdata/stringsdata.h}
\begin{minted}[frame=lines, linenos, fontsize=\large]{c++}
#ifndef STRINGSDATA_H
#define STRINGSDATA_H

#include <cstddef>
#include <string>

struct StringsData 
{
    size_t size;
    std::string *ptr;

    StringsData(size_t size, std::string *ptr);
};

#endif // STRINGSDATA_H

\end{minted}
\subsection*{/stringsdata/stringsdata1.cpp}
\begin{minted}[frame=lines, linenos, fontsize=\large]{c++}
#include "stringsdata.ih"

StringsData::StringsData(size_t size, string *ptr)
    :
        size(size),
        ptr(ptr)
{}

\end{minted}


\section*{Exercise 47}
\subsection*{/main.ih}
\begin{minted}[frame=lines, linenos, fontsize=\large]{c++}
#include "../45-46/strings/strings.h"
#include "../45-46/filter/filter.h"
#include <string>

extern char **environ;

using namespace std;


void string_swap(Strings &string1, Strings &string2);

\end{minted}
\subsection*{/main.cpp}
\begin{minted}[frame=lines, linenos, fontsize=\large]{c++}
#include "main.ih"

int main()
{
    int argc = 3;
    char *argv[] = {"hoi", "wat", "niet"};
    Strings strings1{environ};
    Strings strings2{argc, argv};
    string_swap(strings1, strings2);
    Filter filter1{strings1};
    Filter filter2{strings2};
    filter1.display();
    filter2.display();
}
\end{minted}
\subsection*{/string\_swap.cpp}
\begin{minted}[frame=lines, linenos, fontsize=\large]{c++}
#include "main.ih"

void string_swap(Strings &strings1, Strings &strings2)
{
    
}
\end{minted}

\section*{Exercise 48}
1
[luuk@arch_laptop 48]$ tree
.
├── answer.txt
├── CLASSES
├── data
│   ├── data.h
│   ├── data.ih
│   ├── display.cc
│   └── read.cc
├── data.in
├── icmconf
├── main.cc
└── main.ih

1 directory, 10 files
[luuk@arch_laptop 48]$ cd data
[luuk@arch_laptop data]$ g++ -c *.cc 
[luuk@arch_laptop data]$ ar rvs libdata.a *.o
ar: creating libdata.a
a - display.o
a - read.o
[luuk@arch_laptop data]$ cd ..
[luuk@arch_laptop 48]$ g++ -c main.cc 
[luuk@arch_laptop 48]$ g++ main.o data/libdata.a
[luuk@arch_laptop 48]$ ./program
12
Object 1: value is: 12

2. If I uncomment all occurrences of d_data, we indeed get the following
[luuk@arch_laptop 48]$ cd  data
[luuk@arch_laptop data]$ g++ -c *.cc
[luuk@arch_laptop data]$ ar rvs libdata.a *.o
ar: creating libdata.a
a - display.o
a - read.o
[luuk@arch_laptop data]$ cd ..
[luuk@arch_laptop data]$ g++ main.o data/libdata.a -o program
[luuk@arch_laptop 48]$ ./program
Segmentation fault (core dumped)

Now we migrate to a pimpl implementation.

<code here>

[luuk@arch_laptop 48]$ tree
.
├── answer.txt
├── CLASSES
├── data
│   ├── data1.cc
│   ├── data.h
│   ├── data.ih
│   ├── dataimpl
│   │   ├── dataimpl.h
│   │   ├── dataimpl.ih
│   │   ├── display.cc
│   │   └── read.cc
│   ├── display.cc
│   └── read.cc
├── data.in
├── icmconf
├── main.cc
├── main.ih
└── program

2 directories, 16 files
[luuk@arch_laptop 48]$ cd data
[luuk@arch_laptop data]$ cd dataimpl/
[luuk@arch_laptop dataimpl]$ g++ -c *.cc
[luuk@arch_laptop dataimpl]$ cd ..
[luuk@arch_laptop data]$ g++ -c *.cc
[luuk@arch_laptop data]$ ar rvs libdata.a *.o dataimpl/*.o
ar: creating libdata.a
a - data1.o
a - display.o
a - read.o
a - dataimpl/display.o
a - dataimpl/read.o
[luuk@arch_laptop data]$ cd ..
[luuk@arch_laptop 48]$ g++ -c main.cc
[luuk@arch_laptop 48]$ g++ main.o data/libdata.a -o program
[luuk@arch_laptop 48]$ ./program
12
Object 1: value is: 12

Now comment in d_data.

[luuk@arch_laptop 48]$ cd data/dataimpl/
[luuk@arch_laptop dataimpl]$ g++ -c *.cc
[luuk@arch_laptop dataimpl]$ cd ..
[luuk@arch_laptop data]$ g++ -c *.cc
[luuk@arch_laptop data]$ ar rvs libdata.a *.o dataimpl/*.o
r - data1.o
r - display.o
r - read.o
r - dataimpl/display.o
r - dataimpl/read.o
[luuk@arch_laptop data]$ cd ..
[luuk@arch_laptop 48]$ g++ main.o data/libdata.a -o program
[luuk@arch_laptop 48]$ ./program
12
Object 1: value is: 12

\subsection*{/main.ih}
\begin{minted}[frame=lines, linenos, fontsize=\large]{c++}
#include <iostream>
#include <string>

#include "data/data.h"

using namespace std;

\end{minted}
\subsection*{/data/data.ih}
\begin{minted}[frame=lines, linenos, fontsize=\large]{c++}
#include "data.h"

#include <iostream>

using namespace std;


\end{minted}
\subsection*{/data/data.h}
\begin{minted}[frame=lines, linenos, fontsize=\large]{c++}
#ifndef INCLUDED_DATA_
#define INCLUDED_DATA_

#include "dataimpl/dataimpl.h"

class Data
{
    DataImpl *d_data_impl;

    public:
        Data();
        bool read();
        void display() const; 
};
        
#endif

\end{minted}
\subsection*{/data/dataimpl/dataimpl.ih}
\begin{minted}[frame=lines, linenos, fontsize=\large]{c++}
#include "dataimpl.h"
#include <iostream>

using namespace std;

\end{minted}
\subsection*{/data/dataimpl/dataimpl.h}
\begin{minted}[frame=lines, linenos, fontsize=\large]{c++}
#ifndef DATAIMPL_H
#define DATAIMPL_H

#include <string>

class DataImpl 
{
    // std::string d_text;
    int d_value = 0;

    public:
        bool read();
        void display() const; 
};

#endif // DATAIMPL_H

\end{minted}




\end{document}
